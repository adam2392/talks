\documentclass{amsart}
\usepackage{hyperref}
\usepackage{enumerate}
\renewcommand{\labelitemii}{$\circ$}
\hypersetup{
  colorlinks=true
  }

%%% BEGIN DOCUMENT
\begin{document}

\author{Participants of the Global Brain Workshop$^1$}
\title{Grand Challenges for the Global Brain Sciences}
\date{}
\maketitle

\footnotetext[1]{See \href{http://brainx.io}{http://brainx.io} for a complete list}
Understanding the brain and curing it are among the most
exciting challenges of our time. Consequently, national,
transnational, and private parties are investing billions
of dollars (USD). To efficiently join forces, Global 
Brain Workshop 2016 was hosted at 
Johns~Hopkins~University's Kavli Neuroscience Discovery
Institute on April 7--8. A second workshop, Open 
Data Ecosystem in Neuroscience will happen July 25--26 
in DC to continue the discussion specifically about 
computational challenges and opportunities. A third conference, Coordinating Global Brain Projects, will take
place in New York City on September 19th in association
with the United Nations General Assembly. So vast are
both the challenges and the opportunities that global
coordination is crucial. \\ \\ 

To find ways of synergistically studying the brain, the kick­off
workshop welcomed over 60 scientists, representing 12 different
countries and a wide range of subdisciplines. They were joined by 15
observers from various national and international funding
organizations. Participants were engaged weeks before the conference
and charged with coming up with ambitious projects that are both
feasible and internationally inclusive, on par with the
International Space Station (i.e., worthy of a global, decade--long
effort).  Over the course of 36 hours, scientists discussed, debated, and
gathered feedback, ultimately proposing several ``grand challenges
for global brain sciences.'' The workshop was covered in 
\href{http://science.sciencemag.org/content/352/6283/277}{media piece}
in Science April 15, 2016. The newly established working groups have
continued refining ideas, incorporating feedback from other participants
and interested scientists who were unable to participate to compile a
list of meaningful challenges.


The group began with 60+ ideas, each forged independently by one of the
scientific participants. Each participant proposed a unique challenge 
that was designed to meet the following desiderata: \\

\begin{enumerate}[1.] 
\item \textit{Significant}: it will yield tangible societal, economic, 
	and medical benefits to the world.

\item \textit{Feasible}: it can achieve major milestones within 10 years 
	given existing funding opportunities.

\item \textit{Inclusive}:  nations throughout the world can meaningfully 
  contribute to and benefit from each challenge, and the collection of 
  challenges are collectively scientifically diverse. \\ 
\end{enumerate}


Interestingly, a lot of the proposed ideas were similar to one another
and others were complementary. This allowed the group to converge on
three grand challenges for global brain sciences, each depending on a
common universal resource. \\ \\


\begin{center}
\large \bf Challenge 1: What makes our brains unique?
\vspace{6pt}
\end{center}


Both within and across species, brain structure is known to exhibit
significant variability across many orders of magnitude in 
scale --- \textit{including anatomy, biochemistry, connectivity, development, and gene
expression} (ABCDE). It remains mysterious how and why the nervous system
tightly regulates certain properties, while allowing others to vary.
Understanding the design principles governing variability may hold the
key to understanding intelligence and subjective experience, as well as
the influence of variability on health and function. \\ \\ 


The grand challenge {\bf --- Anatomical Neurocartography --- a global project to 
  coordinate obtaining comprehensive multiscale maps of the ABCDE’s of 
  multiple brains from multiple species using multiple
  cognitive and mental health disease models.}
Within a decade, we expect to have addressed this challenge in brains
including but not limited to Drosophila, Zebrafish, Mouse, and Marmoset,
and to have developed tools to conduct massive neurocartographic
analyses. The result will be a state-of-the-art ``Virtual NeuroZoo'' with
fully annotated data and analytic tools for analysis and discovery. This
virtual NeuroZoo can be utilized by neuroscientists and citizens alike,
both as a reference and for educational materials. By incorporating
disease models, we explicitly link this challenge with the third
challenge. \\ \\ 


\begin{center}
\large \bf Challenge 2: How does the brain solve the complex 
computational problems of intelligence?
\vspace{6pt}
\end{center}


Brains remain the most computationally advanced machines for a large
array of cognitive tasks --- whether navigating hazardous terrain,
translating languages, conducting surgery, or recognizing emotional
states --- despite the fact that modern computers can utilize millions of
training samples, megawatts of power, and tons of hardware. While the
ABCDEs establish the ``wetware'' upon which our brains can solve such
computations, to understand the mechanisms we need to measure,
manipulate, and model neural activity simultaneously across many
spatiotemporal resolutions and scales --- including wearables, embedded
sensors, and actuators --- while animals are exhibiting complex ecological
behaviors in naturalistic environments.

The grand challenge {\bf --- Functional Neurocartography --- is a global project
to coordinate identifying how the multiscale distributed components of
neural architecture orchestrate complex behavior in naturalistic
environments.}  The challenge differs from previous efforts in three key
ways. First, it requires studying animals in complex environments and
behaviors, ideally resembles zoos or even more natural habitats. Second,
it requires coordinated attacks at many different scales by many
different investigators while the animals are performing the same
complex behaviors. We foresee groups of 20--30 investigators all
operating together on shared
data and experimental design. Third, the richness of the mental
repertoire of cognition suggests that deciphering its codes will require
\textit{many parallel investigations} to uncover different facets of brain
function. These experiments in turn will produce multiscale models of
neural systems with the potential to accomplish computational tasks that
no current computer system can perform. Mechanistic studies will help to
ask how perturbations of those systems lead to aberrant function,
thereby linking this challenge with the next one. \\ \\ 

\begin{center}
\large \bf Challenge 3: How can we augment clinical decision­making to
prevent disease and restore brain function? 
\vspace{6pt}
\end{center}


Psychiatric and neurological illnesses levy enormous burdens upon
humanity: impairment, suffering, financial costs, and loss of
productivity. Despite a growing awareness of the challenges, clinicians
consistently battle the lack of objective tests to guide clinical
decision­making (e.g., diagnosis, selection of treatments, prognosis).
Compounding theses limitations are societal stigmas regarding mental
illness that increase the suffering of patients and their families. The
ABCDEs of neurobiological variability, coupled with multiscale
mechanistic models of cognition will provide new approaches to
neurobiologically informed clinical decision making. \\ \\ 


The grand challenge {\bf --- Medical Neurocartography --- a global project to
coordinate, acquire, and model both brain and behavioral data for
brain­based disorders with the goal of augmenting clinical
decision­making with neural mechanisms of dysfunction.}  These datasets,
and the tools developed to explore and discover novel treatment
therapies will be the foundation upon which the next decades of
experiments and clinical decisions will be based. The distributed and
multimodal nature of these datasets further motivate the need for an
all­purpose computational platform, upon which models of disease can be
developed, deployed, tested, and refined. \\ \\ 



\begin{center}
\large \bf A Universal Resource: The International Brain Station
\vspace{6pt}
\end{center}


All three of the above grand challenges for global brain sciences
contain within them severe methodological challenges, including both
technological and computational. The technological developments required
for each of the challenges are essentially non­overlapping. By contrast,
regardless of the nature of the scientific questions or data modalities
involved, each project will require computational capabilities including
collecting, storing, exploring, analyzing, modeling, and discovering
data. Although neuroscience has developed a large number of
computational tools to deal with existing datasets, the datasets
proposed here bring with them a whole suite of new challenges.
The International Brain Station (TIBS) is a comprehensive computational
platform, deployed in the cloud, that will provide Web­services for all
the current ``pain points'' in daily neuroscience practice associated with
big data. TIBS will realize a new era of brain sciences, one in which
the bottlenecks of discovery transition away from data collection and
processing to data enriching
exploring, and modeling. While science has always benefited from
standing on the shoulders of giants, TIBS \textit{will enable science to stand
on the shoulders of everyone.}  Today, essentially every practicing
neuroscientist’s productivity is limited due to computational resources,
access to data or algorithms, or struggling with determining which data
and algorithms are best suited to answer the most pressing questions of
our generation. TIBS will create a future where those limitations will
feel as archaic as fitting the data with paper and pencil feels today.
\\ \\ 


We have identified six stages of science that utilize computational
resources. For each, we have identified the pain points for typical
practicing neuroscientists and have designed a system to minimize those
pain­points as effectively (i.e., easily and inexpensively) as possible.

\begin{enumerate}[1.]
  \item {\bf Collection} The first necessary step for any brain
    investigation, can be tedious and tiresome, requiring manually and
    locally running the measurement devices. \textit{Data Collection
    Dashboards} will directly interface with the machines, to provide live status
    summaries, and enable remote control as appropriate. \\ 

  \item {\bf Storage} The data must be deposited somewhere, and as it
    gets larger and more complicated, neuroscientists are developing
    custom data management solutions. \textit{Multimodal Data Store} will
    interface with the data store to organize data in a fashion readily
    amenable for access and exploration with version control. \\ 

  \item {\bf Analysis} The raw data is often noisy, fragmented, and
    otherwise messy. \textit{Data Analysis Pipelines} will convert raw messy
    data to clean data, automatically and on demand, by pooling the best
    algorithms and implementations from around the world to make them
    readily available for everyone to better explore and model,
    including exhausting quality control on both data and algorithms. \\ 

  \item {\bf Exploration} Whether your own data or shared reference
    datasets, we want access to the data from anywhere at anytime. 
    \textit{Interactive Data Explorers and Notebooks} will dynamically pull both
    raw and processed data to enable navigating and exploring the data
    in augmented reality environments, mobile devices, and other
    devices. \\ 

  \item {\bf Modeling} The final step, prior to suggesting a new
    experiment, is modeling and synthesizing the data. Establishing
    efficient integration of data and models comes with a dual
    challenge. Both datasets and models will be contributed according to
    a set of community driven specifications, such that \textit{Data Modeling
    Bots} will automatically fit existing models to new data, and fit new
    models to existing data, significantly tightening links between
    models and data, to accelerate neuroscientific discovery. \\ 
    
  \item {\bf Discovery} To be useful, neuroscientists of all types must
    be able to quickly and easily find data appropriately matched to
    their scientific goals. To that end, a \textit{Metadata Query
    Service} will link to a community established set of metadata fields, 
    to enable both searching and juxtaposing similar and disparate
    datasets. \\ 
    
  \item {\bf Education} the above steps represent a transformation in
    the practice of brain sciences, and therefore, its success will
    depend on extensive training and educational material, designed from
    the ground up to be culturally sensitive and universally accessible.
\end{enumerate}



Crucial to the success of TIBS will be tight integration with each of the
three above motivating grand challenges, and the neuroscientists engaged
in pursuing them. Moreover, community involvement and feedback at each
stage will be key. TIBS will be entirely open source and community
driven: both data and tools will be voted on and reviewed, forming the
meritocracy the science deserves. In this sense, TIBS will be the most
inclusive aspect of all of the proposed grand challenges, because the
barrier to enter will be as low as possible. \\ \\ 


Common Threads \\ \\ 


In addition to TIBS linking all three grand challenges of global brain
sciences, there are two key threads to this proposal that differentiate
it from previous efforts. First, the data collection efforts will be
multimodal and multiscale. Historically most neuroscientists focus on a
small number experimental modalities and scales (e.g., extracellular
electrophysiology), answering the above challenges satisfactorily will
require {\bf many groups jointly investigating} both across modalities and
scales. Because designing and completing these experiments for even a
single modality and scale will be a monumental task, coordinating across
brain scientists will be essential. Second, the data and tools will be
designed and developed at the outset to be {\bf reference, rather than
personal.} Collecting reference datasets, and writing reference
implementations is different than standard academic practice. In
particular, it requires considerable investment in training and support
to facilitate other people effectively using the provided resources.
Thus, to be effective, investigators will have to commit a sizable
fraction of effort to supporting those activities. \\ \\ 


Societal Considerations \\ \\ 


Each nation affords different societal opportunities and restrictions,
due to ethical, policy, and cultural considerations. Because these grand
challenges are inherently inclusive, manifesting them will require
understanding and mitigating issues that will arise in any
cross­cultural endeavor. Indeed, addressing the vast diversity of
partnerships in such an endeavor is a challenge in itself. We therefore
recommend the following. First, form a {\bf cultural sensitivity committee}
to consider and investigate potentially sensitive issues. Second,
bolstered by their research, establish {\bf cross­cultural collaboration
education materials}, including written guidelines and videos, which will
be recommended to all participating scientists. Third, to deepen the
understanding of transnational collaborations, develop {\bf trainee exchange
programs}, where participating trainees will spend six months to a year
working and training in a
foreign country. This will also facilitate cross­cultural knowledge
dissemination and fertilization. Fourth, require {\bf frequent assessments}
to ensure maintenance of cultural sensitivities. These assessments will
feedback into the educational materials, and be used to modify the
exchange programs, thereby facilitating consistent considerations even
as the goals and individuals change over the time­course of these grand
challenges. \\ \\ 


Next Steps \\ \\ 


Crucial to the success of this endeavor is a sequence of actionable
steps that the community can follow. Because we are not actually
proposing any additional funding, realizing the eventual goals of these
grand challenges will rely on marshalling existing funds. Due to the
incoming leadership changes, both on national and transnational levels,
time for acting is of the essence. Therefore, we have taken the
following steps:

\begin{itemize}
  \item We have created a webpage, \href{http://brainx.io}{http://brainx.io}, containing the
    following:
    \begin{itemize}
      \item the current working version of this document, and
      \item a list of all scientific participants and observers who
        attended the original brainstorming meeting leading to this document.
    \end{itemize}
  \item We will provide (possibly via 
        \href{https://neurostars.org/}{https://neurostars.org/}):
    \begin{itemize}
      \item a platform for discussion and improving the ideas presented
        herein, 
      \item a forum for ``teaming,'' that is, openly discussing these
        ideas to develop more
        specific and concrete plans for individual teams to pursue, and

      \item a list of potential funding mechanisms that might support
        these endeavors, such as NSF’s 
        \href{http://www.nsf.gov/pubs/2016/nsf16569/nsf16569.htm}{NeuroNex} 
        and their
        \href{http://www.nsf.gov/pubs/2016/nsf16076/nsf16076.jsp}{NSF 16­076}.
    \end{itemize}
  \item The following conferences will continue to refine the ideas
    discussed herein:
    \begin{itemize}
      \item Open Data Ecosystem for the Neurosciences on July 25­26,
        2016
      \item Coordinating Global Brain Projects at Rockefeller on
        September 19, 2016, which will be publicly broadcast
      \item A planned United Nations General Assembly Side Event, also
        in September.
    \end{itemize}
\end{itemize}

We encourage anybody who feels inspired by this document to join the
discussion, begin teaming activities, and get in touch with program
officers with your ideas.

\end{document}
